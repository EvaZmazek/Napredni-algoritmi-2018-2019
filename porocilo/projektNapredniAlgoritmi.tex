\documentclass[12pt,a4paper,twoside]{article}
\usepackage[utf8]{inputenc}  % pravilno razpoznavanje unicode znakov

\usepackage[slovene]{babel}  % slovenščina
\usepackage[T1]{fontenc}     % naprednejše kodiranje fonta
\usepackage{amsmath,amssymb,amsfonts,amsthm} % matematični paketi
\usepackage[dvipsnames,usenames]{color} % barve
\usepackage{graphicx}     % za slike
\usepackage{emptypage}    % prazne strani so neoštevilčene, ampak so štete
\usepackage{tikz}

% oblika strani
\usepackage[
  top=3cm,
  bottom=3cm,
  inner=3.5cm,      % margini za dvostransko tiskanje
  outer=2.5cm,
  footskip=40pt     % pozicija številke strani
]{geometry}

% VEČ ZANIMIVIH PAKETOV
% \usepackage{array}      % več možnosti za tabele
% \usepackage[list=true,listformat=simple]{subcaption}  % več kot ena slika na figure, omogoči slika 1a, slika 1b
% \usepackage[all]{xy}    % diagrami
% \usepackage{doi}        % za clickable DOI entrye v bibliografiji
% \usepackage{enumerate}     % več možnosti za sezname

% Za barvanje source kode
% \usepackage{minted}
% \renewcommand\listingscaption{Program}

% Za pisanje psevdokode
% \usepackage{algpseudocode}  % za psevdokodo
% \usepackage{algorithm}
% \floatname{algorithm}{Algoritem}
% \renewcommand{\listalgorithmname}{Kazalo algoritmov}

% DRUGI TVOJI PAKETI:
% tukaj

\setlength{\overfullrule}{50pt} % označi predlogo vrstico
\pagestyle{plain}               % samo številka strani na dnu, nobene glave / noge

% ukazi za matematična okolja
\theoremstyle{definition} % tekst napisan pokončno
\newtheorem{definicija}{Definicija}[section]
\newtheorem{primer}[definicija]{Primer}
\newtheorem{opomba}[definicija]{Opomba}
\newtheorem{aksiom}{Aksiom}

\theoremstyle{plain} % tekst napisan poševno
\newtheorem{lema}[definicija]{Lema}
\newtheorem{izrek}[definicija]{Izrek}
\newtheorem{trditev}[definicija]{Trditev}
\newtheorem{posledica}[definicija]{Posledica}

\numberwithin{equation}{section}  % števec za enačbe zgleda kot (2.7) in se resetira v vsakem poglavju

% Matematični ukazi
\newcommand{\R}{\mathbb R}
\newcommand{\N}{\mathbb N}
\newcommand{\Z}{\mathbb Z}
%\renewcommand{\C}{\mathbb C}
\newcommand{\Q}{\mathbb Q}

% \DeclareMathOperator{\tr}{tr}  % morda potrebuješ operator za sled ali kaj drugega?

% bold matematika znotraj \textbf{ }, tudi v naslovih, kot \omega spodaj
%\makeatletter \g@addto@macro\bfseries{\boldmath} \makeatother
%
%% Poimenuj kazalo slik kot ``Kazalo slik'' in ne ``Slike''
%\addto\captionsslovene{
%  \renewcommand{\listfigurename}{Kazalo slik}%
%}


%%%%%%%%%%%%%%%%%%%%%%%%%%%%%%%%%%%%%%%%%%
%%%%%%           DOCUMENT           %%%%%%
%%%%%%%%%%%%%%%%%%%%%%%%%%%%%%%%%%%%%%%%%%

\begin{document}

\thispagestyle{empty} % ampak na prvi strani ni številke



\begin{titlepage}\centering
\noindent{\large
Univerza v Mariboru \\[1mm]
Fakulteta za naravoslovje in matematiko\\[5mm]}
\vspace*{\fill}
{\LARGE Linearni problem prevoza} \\[1cm]
\large Eva Zmazek
\vspace*{\fill}
\vfill
\noindent{\large Maribor, 2019}
\end{titlepage}

\cleardoublepage


\section{Opredelitev problema linearnega prevoza}

\section{Problem linearnega prevoza}

Iz skladišč želimo prepeljati večje količine izdelka na različne lokacije (trgovine). Imamo $n$ skladišč in $k$ trgovin. Skladišča imajo vsaka svojo zalogo izdelka, vsaka trgovina pa želi prejeti določeno količino izdelka. Za prevoz izdelka iz $i$-tega skladišča v $j$-to trgovino imamo v naprej določeno ceno prevoza. Izračunati je potrbno, kolikšen je minimalen strošek prevoza, če želimo izpolniti vse zahteve trgovin. Problem je \textbf{linearen}, če strošek prevoza iz skladišča v trgovino ni odvisen od količine, ko jo prevozimo, torej je cena prevoza na izdelek konstanta. \\

Definirajmo:
\begin{itemize}
\item $sour(i)$ \dots \textbf{zaloga} izdelka v $i$-tem skladišču
\item $dest(j)$ \dots \textbf{zahteva} $j$-te trgovine
\item $cost(i,j)$ \dots \textbf{cena} prevoza enega izdelka iz $i$-tega skladišča v $j$-to trgovino
\item $x_{i,j}$ \dots \textbf{količina prevoženega izdelka} iz $i$-tega skladišča v $j$-to trgovino (pri konkretnem prevozu)
\end{itemize}

\noindent Glede na zgornje oznake lahko problem linearnega prevoza zapišemo kot:
$$\min \sum\limits_{i=1}^{n} \sum\limits_{j=1}^{k} cost(i,j) \cdot x_{i,j}$$

\noindent Pri reševanju problema moramo upoštevati še naslednje pogoje:
\begin{itemize}
\item $\sum\limits_{j=1}^{k} x_{i,j} \leq sour(i);~ i=1,2,\dots,n$ \\ (iz $i$-tega skladišča ne moremo odpeljati večje kotličine, kot je zaloga skladišča)
\item $\sum\limits_{i=1}^{n} x_{i,j} \geq dest(j);~ j=1,2,\dots,k$ \\ (v $j$-to trgovino moramo pripeljati najmanj zahtevano količino trgovine)
\item $x_{i,j} \geq 0;~ i=1,2, \dots n ~ \text{in} ~ j=1,2, \dots, k$ \\
(prevažamo vedno iz skladišča v trgovino in ne obratno)
\end{itemize}

\newpage
\subsection{Uravnotežen problem linearnega prevoza}

Če velja $\sum\limits_{i=1}^{k} sour(i) = \sum\limits_{j=1}^{n}$, pravimo da je problem linearnega prevoza \textbf{uravnotežen}. \\

\noindent Posledično veljata tudi naslednji enakosti:
\begin{itemize}
\item $\sum\limits_{j=1}^{k} x_{i,j} = sour(i);~ i=1,2,\dots,n$ 
\item $\sum\limits_{i=1}^{n} x_{i,j} = dest(j);~ j=1,2,\dots,k$
\end{itemize}

 \noindent Znotraj tega projekta bomo obravnavali samo uravnotežene problem linearnega prevoza.

\subsection{Uprozoritveni problem}
$sour = [1, 0, 0, 0, 3]$
$dest = [3, 1, 0]$
$$ cost = 
\begin{tabular}{c c c c c | r}
 1 & 1 & 2 & 1 & 1 & 3\\
 3 & 2 & 4 & 1 & 1 & 1\\
 1 & 1 & 1 & 1 & 10& 0\\
 \hline
  1 & 0 & 0 & 0 & 3 & 4\\
\end{tabular}
$$
$$ v =
\begin{tabular}{c c c c c}
 1 & 0 & 0 & 0 & 2\\
 0 & 0 & 0 & 0 & 1\\
 0 & 0 & 0 & 0 & 0\\
 \end{tabular}
 $$
 Celoten optimalen strošek je $4$.
\subsection{Celoštevilski uravnotežen problem linearnega prevoza}

Če za vsako skladišče $i$ velja, da je $sour(i)$ naravno število in za vsako trgovino $j$ velja, da je $dest(j)$ naravno število v uravnoteženem problemu linearnega prevoza, potem je tudi optimalna rešitev uravnoteženega problema linearnega prevoza celoštevilska (to pomeni, da so vrednosti $x_{i,j}$ naravna števila. Velja tudi, da je število pozitivnih (torej različnih od $0$) vrednosti $x_{i,j}$ največ $k+n-1$.

\section{Klasični genetski algoritem}

Da je algoritem klasični pomeni, da so kromosomi pradstavljeni z dvojiškim zapisom. To pomeni, da so možne rešitve predstavljene z vektorjem $$(v_1, v_2, v_3, \dots, v_p);~p=n \cdot k \text{,}$$ pri čemer je komponenta vektorja $v_i$  dvojišči vektor $$(w_0^i, w_1^i, \dots, w_s^i) \text{,}$$ ki predstavlja vrednost $x_{j,m}$ za $j= \lfloor \frac{i-1}{k+1} \rfloor$ in $m= (i-1) mod (k+1)$. \\

\noindent Za vsako dopustno rešitev tega problema mora veljati:
\begin{itemize}
\item $v_q \geq 0$ za $q=1, 2, \dots, k \cdot n$,
\item $\sum\limits_{i=c \cdot k + 1}^{c \cdot k + k} v_i = sour(c+1)$ za $c=0, 1, \dots, n-1$,
\item $\sum\limits_{j=m, \text{ korak } k}^{k \cdot n} v_j = dest(m)$ za $m=1, 2, \dots, k$.
\end{itemize}

\subsection{Evaluation function}

Evaluacijska funkcija tega problema je skupna cena prevoza, pri čemer so izpolnjeni vsi zgornji pogoji.

$$eval((v_1, v_2, v_3, \dots, v_p)) = \sum\limits_{i=1}^{p} v_i \cdot cost(j,m),$$

kjer je $j=\lfloor \frac{i-1}{k+1} \rfloor$ in $m=(i-1) mod(k+1)$.

\subsection{Genetic operators}

Mutacija in križanje za klasični genetični algoritem naletita na mnoge probleme, ki privedejo do ugotovitve, da ta implementacija ni najbolj primerna.

\section{GEN1 - Izboljšava vektorske reprezentacije in \\ njena implementacija}

V tem poglavju bomo opisali implementacijo prvega genetskega algoritma, ki ga bomo označevali z oznako $GEN1$.

\subsection{Inicializacija-inicialization}

Če želimo dobiti začetno populacijo, moramo zgenerirati nekaj dopustnih rešitev, ki zadoščajo zgornjim pogojem. S funckijo $inicialization$. 

\noindent Rešitve podajamo v obliki permitacije, kjer permutacija določa zaporedje izbire celice v matriki, po kateres se potem izračuna inicializacija rešitve.

\subsection{Genetic operators}

Za konstrukcijo nove generacije uporabimo eno izmed možnosti inverzije (obrata), mutacije ali križanja. Inverzija permutacijo, ki predstavlja rešitev samo obrne (torej izbiramo v obratnem vrstnem redu, mutacije zamenja dva elementa permutacije, križanje pa izbere del prvega starša (prve permutacije) ter preostali del zapolni z mankajočimi elementi v zaporedju, kot ga določa drugi starš (druga permutacija).

\section{GEN-2}

V tem algoritmu rešitve predstavljamo v obliki matrike, kar se zdi bolj naravno glede na problem.

\noindent Za vsako dopustno rešitev tega problema mora veljati:
\begin{itemize}
\item $v_{i,j} \geq 0$ za $i=1, 2, \dots, k;~ j=1,2, \dots, n$,
\item $\sum\limits_{i=1}^{k} v_{i,j} = dest(j)$ za $j=1, \dots, n$,
\item $\sum\limits_{j=1}^{n} v_{i,j}= sour(i)$ za $i=1, \dots, k$.
\end{itemize}

\subsection{Evaluation function}

$$eval(v_{i,j}) = \sum\limits_{i=1}^{k} \sum\limits_{j=1}^{n} v_{i,j} \cdot cost(i,j)$$

\subsection{Genetic operators}

\subsubsection{Mutation}


\end{document}
